\chapter{Related Work}
\label{sec:rel_work}



% The goal of this work is to provide a gemetric and semantic traversability map from a stereo image pairs. 

Geometric traversability. 

A number of works have been performed. Fankhuser have built a geometric traversability estimation that levereages 
the slope: meaning the angle of the obstacles surfaces normal and the roughness: 
meaning the difference between the elvation estimation and the smoothed elevation estimation (reference fankhauser) 
In .. they use ...


Semantic mapping.
Semantic Mapping has shown to be a critical component to estimate the traversability to ensure a safe  autonomous navigation of ground robots. In \citep{Rosinol2020,Paz2020a, Gan2021a} (13,10, 17,24,23, 22,19,12,10,7) the authors aim to embed semantic information into a 3D voxelized space. 
However, the high dimensionality of the map has shown to be very memory consuming and computationally intensive.

%TODO: further describe how they fuse the points into the map. Maybe put it into three categories: 

averaging, voting scheme (25,10) use an averaging scheme to build the map from the 3D and semantic information.
In (24,13) they build the map through a Bayesian framework and thus infer the measurements into the map over time.
In (13) they exploit inter-layer correlations to have a deeper understanding of the environment.
 In (23,22,19,7) Conditional Random Fields (CRF) or Bayes spatial kernels (actually these are continuous maps) are used to relax the independent grid assumption and thus regularizing and smooth the map.
 (12,14) adopt a method to create continuous maps by using Gaussian Processes to fuse the geometrical and the semantic information into the map. This approach however is computationally very intensive ($N^3$).


In \citep{Cartillier2021a, } (26,27, 18,16) the authors use a 2.5D map that stores the elevation information of the map and include also the semantic information of the map.
Semantic approach:
To store semantic information in a map coming from a egocentric remote sensor such as a camera mainly two approaches have been pursued. (citation) perform a semantic segmentation directly on the egocentric image and then project the semantic classes into the map. 
Similarly to our work at MapNet and ... the authors perform a feature extraction on the egocentric image and then project the features into the map. 





Start with least similar and end with most similar approach.




